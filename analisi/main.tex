\documentclass{paper}

\usepackage{amsfonts}
\usepackage{amsmath}

\begin{document}
    \section{Modellazione matematica del problema}

    Sia $n$ il numero di partite. $y_i$ è l'outcome di quella partita.
    $x_i$ è la descrizione dalla partita $i$.

    Sia $p$ il numero di feature.

    Vogliamo indovinare una funzione $f: \mathbb{R}^p \rightarrow \{y_0,y_1\}$

    $y_0$ è il valore nel caso la partita è stata persa, $y_1$ altrimenti.

    Nel nostro dataset abbiamo che $y_i = f(x_i) + \varepsilon$. dove $\varepsilon \sim \mathcal{N}(0,\sigma^2)$.

    \begin{equation}
        \mathcal{D} = \{ (x_i, y_i) \mid i \in [1,n] \land x_i \in \mathbb{R}^p \land y_i \in \{y_0, y_1\}\}
    \end{equation}


    Com'è composto $x_i$?

    \begin{equation}
        x_i = \text{StatoPoks}_i \circ \text{StatoPoksAdv}_i \circ \text{tipiPoks}_i
    \end{equation}


    $\text{StatoPoks}_i$ e $\text{StatoPoksAdv}_i$ hanno diverse feature.

    Queste feature sono la media dei vari pokemon.

    Sia $\text{StatoPok}_i^j$ lo stato di un singolo pokemon della partita $i$ del pokemon $j \in \{1, \dots, 6\}$.

    \begin{equation}
        \text{StatoPoks}_i^f = \frac{1}{6} \sum_{j \in \{1, \dots, 6\}} StatoPok_i^{j,f} \mid f \in \mathcal{F}
    \end{equation}


    Dove $\mathcal{F}$ è l'insieme delle feature di un pokemon.

    Nota bene: se la vita di un pokemon è 0, allora tutte le altre feature devono essere 0.

    Speculare è per l'avversario.

    \begin{equation}
        \text{StatoPoksAdv}_i^f = \frac{1}{6} \sum_{j \in \{1, \dots, 6\}} StatoPokAdv_i^{j,f} \mid f \in \mathcal{F}
    \end{equation}

    L'unica differenza è che potremmo non poter osservare tutti i pokemon dell'avversario. Se non conosciamo il $j$-esimo pokemon allora
    possiamo inserire il pokemon medio. Ovvero la media di tutte le feature di tutti i pokemon conosciuti che possono capitare.

    $\text{tipiPoks}_i$ è il hot one encode dei tipi dei pokemon di una partita.

    Per ogni pokemon possiamo fare il hot one encode dei tipi di quel pokemon. Poi vengono sommati a tutta la squadra e sottratti
    al hot one encode dell'avversario. Se un pokemon è morto allora non viene considerato.% TODO: formalizza

\end{document}